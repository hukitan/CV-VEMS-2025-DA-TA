\documentclass[10pt, a4paper]{article}

% Packages:
\usepackage[
    ignoreheadfoot, % set margins without considering header and footer
    top=2 cm, % seperation between body and page edge from the top
    bottom=2 cm, % seperation between body and page edge from the bottom
    left=2 cm, % seperation between body and page edge from the left
    right=2 cm, % seperation between body and page edge from the right
    footskip=1.0 cm, % seperation between body and footer
    % showframe % for debugging 
]{geometry} % for adjusting page geometry
\usepackage{titlesec} % for customizing section titles
\usepackage{tabularx} % for making tables with fixed width columns
\usepackage{array} % tabularx requires this
\usepackage[dvipsnames]{xcolor} % for coloring text
\definecolor{primaryColor}{RGB}{0, 0, 0} % define primary color
\usepackage{enumitem} % for customizing lists
\usepackage{fontawesome5} % for using icons
\usepackage{amsmath} % for math
\usepackage[
    pdftitle={Elias Mina's CV},
    pdfauthor={Elias Mina},
    pdfcreator={LaTeX with RenderCV},
    colorlinks=true,
    urlcolor=primaryColor
]{hyperref} % for links, metadata and bookmarks
\usepackage[pscoord]{eso-pic} % for floating text on the page
\usepackage{calc} % for calculating lengths
\usepackage{bookmark} % for bookmarks
\usepackage{lastpage} % for getting the total number of pages
\usepackage{changepage} % for one column entries (adjustwidth environment)
\usepackage{paracol} % for two and three column entries
\usepackage{ifthen} % for conditional statements
\usepackage{needspace} % for avoiding page brake right after the section title
\usepackage{iftex} % check if engine is pdflatex, xetex or luatex

% Ensure that generate pdf is machine readable/ATS parsable:
\ifPDFTeX
    \input{glyphtounicode}
    \pdfgentounicode=1
    \usepackage[T1]{fontenc}
    \usepackage[utf8]{inputenc}
    \usepackage{lmodern}
\fi

\usepackage{charter}
\usepackage{accsupp} % for PDF accessibility support
\newcommand{\unselectable}[1]{%
    \BeginAccSupp{method=escape,ActualText={}}#1\EndAccSupp{}%
}
\pdfcompresslevel=9
\pdfobjcompresslevel=9

% Some settings:
\raggedright
\AtBeginEnvironment{adjustwidth}{\partopsep0pt} % remove space before adjustwidth environment
\pagestyle{empty} % no header or footer
\setcounter{secnumdepth}{0} % no section numbering
\setlength{\parindent}{0pt} % no indentation
\setlength{\topskip}{0pt} % no top skip
\setlength{\columnsep}{0.15cm} % set column seperation
\pagenumbering{gobble} % no page numbering

\titleformat{\section}{\needspace{4\baselineskip}\bfseries\large}{}{0pt}{}[\vspace{1pt}\titlerule]

\titlespacing{\section}{
    % left space:
    -1pt
}{
    % top space:
    0.3 cm
}{
    % bottom space:
    0.2 cm
} % section title spacing

\renewcommand\labelitemi{$\vcenter{\hbox{\small$\bullet$}}$} % custom bullet points
\newenvironment{highlights}{
    \begin{itemize}[
        topsep=0.10 cm,
        parsep=0.10 cm,
        partopsep=0pt,
        itemsep=0pt,
        leftmargin=0 cm + 10pt
    ]
}{
    \end{itemize}
} % new environment for highlights


\newenvironment{highlightsforbulletentries}{
    \begin{itemize}[
        topsep=0.10 cm,
        parsep=0.10 cm,
        partopsep=0pt,
        itemsep=0pt,
        leftmargin=10pt
    ]
}{
    \end{itemize}
} % new environment for highlights for bullet entries

\newenvironment{onecolentry}{
    \begin{adjustwidth}{
        0 cm + 0.00001 cm
    }{
        0 cm + 0.00001 cm
    }
}{
    \end{adjustwidth}
} % new environment for one column entries

\newenvironment{twocolentry}[2][]{
    \onecolentry
    \def\secondColumn{#2}
    \setcolumnwidth{\fill, 4.5 cm}
    \begin{paracol}{2}
}{
    \switchcolumn \raggedleft \secondColumn
    \end{paracol}
    \endonecolentry
} % new environment for two column entries

\newenvironment{threecolentry}[3][]{
    \onecolentry
    \def\thirdColumn{#3}
    \setcolumnwidth{, \fill, 4.5 cm}
    \begin{paracol}{3}
    {\raggedright #2} \switchcolumn
}{
    \switchcolumn \raggedleft \thirdColumn
    \end{paracol}
    \endonecolentry
} % new environment for three column entries

\newenvironment{header}{
    \setlength{\topsep}{0pt}\par\kern\topsep\centering\linespread{1.5}
}{
    \par\kern\topsep
} % new environment for the header

\newcommand{\placelastupdatedtext}{% \placetextbox{<horizontal pos>}{<vertical pos>}{<stuff>}
  \AddToShipoutPictureFG*{% Add <stuff> to current page foreground
    \put(
        \LenToUnit{\paperwidth-2 cm-0 cm+0.05cm},
        \LenToUnit{\paperheight-1.0 cm}
    ){\vtop{{\null}\makebox[0pt][c]{
        \small\color{gray}\textit{Last updated in July 2024}\hspace{\widthof{Last updated in July 2024}}
    }}}%
  }%
}%

% save the original href command in a new command:
\let\hrefWithoutArrow\href

% new command for external links:


\begin{document}

    \newcommand{\AND}{\unskip
        \cleaders\copy\ANDbox\hskip\wd\ANDbox
        \ignorespaces
    }
    \newsavebox\ANDbox
    \sbox\ANDbox{$|$}

    \begin{header}
        \fontsize{25 pt}{25 pt}\selectfont V. Elias Mina S.

        \vspace{5 pt}

        \normalsize
        \mbox{Ciudad de México}%
        \kern 5.0 pt%
        \AND%
        \kern 5.0 pt%
        \mbox{\hrefWithoutArrow{mailto:eminasol@outlook.com}{eminasol@outlook.com}}%
        \kern 5.0 pt%
        \AND%
        \kern 5.0 pt%
        \mbox{\hrefWithoutArrow{https://wa.me/525540936066}{+52 55 4093 6066}}%
        \kern 5.0 pt%
        \AND%
        \kern 5.0 pt%
        \mbox{\hrefWithoutArrow{https://eminasol.my.canva.site/}{Portfolio}}%
        \kern 5.0 pt%
        \AND%
        \kern 5.0 pt%
        \mbox{\hrefWithoutArrow{https://www.linkedin.com/in/eminaso/}{linkedin.com/in/eminaso}}%
        \kern 5.0 pt%
        \AND%
        \kern 5.0 pt%
        \mbox{\hrefWithoutArrow{https://github.com/hukitan}{github.com/hukitan}}%
    \end{header}

    \vspace{5 pt - 0.3 cm}


    \section{Perfil}



        
        \begin{onecolentry}
           Analista de datos con experiencia en Python, R, SQL, Excel, Power BI y Tableau, recientemente completada la especialización en análisis de datos en TripleTen. Más de 1,5 años de experiencia práctica en limpieza de datos, visualización y generación de información empresarial. Combina habilidades analíticas con más de 5 años de experiencia en reclutamiento de TI, aprovechando estrategias basadas en datos para optimizar los procesos de contratación. Sólida formación en análisis estadístico, desarrollo de paneles de control y modelado predictivo. Habla inglés y español con fluidez.
           % \href{https://github.com/sinaatalay/rendercv}{RenderCV} is a LaTeX-based CV/resume framework. It allows you to create a high-quality CV or resume as a PDF file from a YAML file, with \textbf{full Markdown syntax support} and \textbf{complete control over the LaTeX code}.
        \end{onecolentry}

        \vspace{0.2 cm}

        %\begin{onecolentry}
           %Bilingual IT recruitment professional with over 5 years of experience, including 3 years specializing in technology recruitment. Known for connecting companies with top talent, driving success for both parties. Key achievements include building testing teams from scratch and supporting market entry for tech companies in Mexico. 

          %  The boilerplate content is taken from \href{https://github.com/dnl-blkv/mcdowell-cv}{here}, where a \textit{clean and tidy CV} pattern is proposed by \textbf{\href{https://www.gayle.com/}{Gayle Laakmann McDowell}}.
      %  \end{onecolentry}


        \section{Proyectos de datos}

    \begin{twocolentry}{
            Proyectos en Tripleten
        }
            \textbf{Especialiazcion en analisis de datos}\end{twocolentry}

        \vspace{0.10 cm}
        \begin{onecolentry}
            \begin{highlights}
                \item Analizar datos de streaming de música online para comparar las preferencias musicales en Springfield y Shelbyville.
                \item Limpiar datos de Instacart y preparar un informe sobre los hábitos de compra de los clientes.
                \item Analizar datos de 500 clientes de Megaline (telecomunicaciones) para determinar qué tarifa prepago (Surf o Ultimate) genera más ingresos.
                \item Analizar datos históricos de videojuegos (reseñas, géneros, plataformas, ventas) para identificar patrones de éxito.
                \item Practicar tareas comunes de ingeniería de software para aumentar las habilidades con los datos mediante la creación de un panel de control en una página web (streamlit) con datos de los últimos 5 años del metro de CDMX.
                \item Analizar los datos de Zuber (empresa de taxis) para identificar los patrones de preferencia de los pasajeros y el impacto de los factores externos.
                \item Analizar los datos sobre visitas, pedidos y gasto en marketing en Showz para optimizar el gasto en marketing.
                \item Analizar los datos de una empresa de comercio electrónico de una prueba A/B y la comprobación de hipótesis. Realizar un análisis de los métodos de priorización ICE \& RICE.
                \item Análisis del embudo de ventas y pruebas A/A/B para evaluar el impacto de un cambio de fuentes en la implementación de una empresa alimentaria.
                \item Desarrollar un panel de control para automatizar el análisis semanal de tendencias de los vídeos de YouTube, clasificando el contenido por región y popularidad en Estados Unidos.
                \item Analizar los datos de clientes de "Model Fitness" para predecir la probabilidad de abandono, identificar los segmentos de usuarios clave y determinar los principales factores que influyen en la retención. Proporcionar recomendaciones para reducir la pérdida de clientes.
                \item Analizar los datos de una de las dos áreas (telecomunicaciones y comercio electrónico), incluyendo el preprocesamiento, el análisis exploratorio de datos, las consultas SQL, la comprobación de hipótesis, el trabajo con métricas empresariales y la preparación de una presentación. Además, crear un panel de control para la visualización.
            \end{highlights}
        \end{onecolentry}


        \vspace{0.2 cm}

    \begin{twocolentry}{
            
        } 
            \textbf{Proyectos Personales}\end{twocolentry}

        \vspace{0.10 cm}
        \begin{onecolentry}
            \begin{highlights}
                \item Realizar un script en R para la puntuación y limpieza de datos de una prueba psicométrica elaborada en Google Forms. Los resultados se escriben en una hoja de Excel. 
                \item Asistencia en el análisis de los resultados de tesis de grado mediante R (estadística descriptiva, modelos lineales, gráficos). Los resultados generados se guardaron en una hoja de cálculo de Excel.
                \item Creación de un script sencillo en Python para analizar mensajes de WhatsApp en un periodo de tiempo determinado (frecuencia de respuesta, tiempo de respuesta, frecuencia de palabras y sus gráficos).
                \item Desarrollo de cuestionarios y análisis de datos psicométricos, ayudando a compañeros con el código para la puntuación de pruebas.
                \item Implementación de soluciones de validación y limpieza de datos (valores perdidos, duplicados).
                \item Propuesta de recomendaciones basadas en datos para la toma de decisiones en las áreas de marketing, finanzas y recursos humanos.
            \end{highlights}
        \end{onecolentry}
        

        \vspace{0.2 cm}
        
\newpage
    \section{Experiencia}

 \begin{twocolentry}{
            %Feb 2025 -- Apr 2025
        }
            \textbf{Sr. IT Recluiter} | CHR IT (Abr-25-Actual); Branchbit (Feb 2025 – Abr 2025); Recruit IT (Mar 2022 – Ene 2025)\end{twocolentry}

        \vspace{0.10 cm}
        \begin{onecolentry}
            \begin{highlights}
               \item Reclutamiento de ciclo completo utilizando un enfoque basado en datos para analizar las tendencias de contratación y optimizar los procesos con KPI y métricas.
                \item Realización de estudios socioeconómicos y verificación de referencias laborales mediante análisis de datos estructurados para garantizar la fiabilidad y el cumplimiento.
                \item Análisis de informes salariales del mercado para proporcionar estimaciones de remuneración para puestos de TI.
                \item Actividades de gestión de la oficina relacionadas con la contratación, incluyendo el análisis de la eficiencia de los procesos, la asignación de recursos y la optimización de la logística.
                \item Coordinación de entrevistas y optimización de procesos mediante el uso de análisis para realizar un seguimiento de los tiempos de respuesta, la experiencia de los candidatos y las tasas de conversión.
                \item Gestión y actualización continua de las bases de datos de candidatos, aplicando técnicas de análisis de datos para mejorar las estrategias de selección y reducir el tiempo de contratación.
                \item Utilización de ATS (sistemas de seguimiento de candidatos) con análisis de datos para el seguimiento de los candidatos, la optimización del embudo de contratación y la medición del rendimiento.
                \item Apoyo en los procedimientos de incorporación, analizando las métricas de rendimiento de los nuevos empleados y las tasas de retención para mejorar el proceso de integración.
            \end{highlights}
        \end{onecolentry}


        \vspace{0.2 cm}

    
 %end of non select 


    \section{Idiomas}
        \begin{samepage}
            \begin{onecolentry}
                \begin{highlightsforbulletentries}
                \item Español (Native)
                \item Inglés (Advanzado)
                \end{highlightsforbulletentries}
            \end{onecolentry}
        \end{samepage}
         
    \section{Conocimiento}
        \begin{samepage}
            \begin{onecolentry}
                \begin{highlightsforbulletentries}
                \item  MS Office (7 años)
                \item  Reclutamiento (5 años)
                \item  Gestión de documentos (3 años)
                \item  Formación (2 años)
                \item  Procesos de RR. HH. (2 años)
                \item Análisis de datos: (2 años)
                \begin{itemize}
                    \item  Herramientas: Python, R, SQL, Excel, Power BI, Tableau
                    \item Técnicas: Análisis exploratorio de datos (EDA), visualización de datos, limpieza de datos, creación de paneles de control, web scraping
                \end{itemize}
                \end{highlightsforbulletentries}
            \end{onecolentry}
        \end{samepage} 
%\newpage

    \section{Educación}
        \begin{samepage}
        \begin{twocolentry}{      Sept 2014 – Nov 2018 }
            \textbf{UNAM}, Lic. Psicología  \end{twocolentry}
                \begin{onecolentry}
            \begin{highlights}
                % \item  
                \item \textbf{Area:} Neurociencia, Estadística, Análisis de datos, Método científico
            \end{highlights}
        \end{onecolentry}\textbf{}
        % nuevo exp
        \vspace{0.10 cm}


        \begin{twocolentry}{    Agosto 2024 – Marzo 2025   }
            \textbf{TripleTen}, Especializacion en analisis de datos\end{twocolentry}
        \begin{onecolentry}
            \begin{highlights}
            % \item  
                \item \textbf{Conocimientos:} Python, Estadística, Análisis de datos, Ingeniería de software, SQL, Web scraping.
            \end{highlights}
        \end{onecolentry}\textbf{}
        \vspace{0.10 cm}
        \end{samepage} 

        
    \section{Cursos y actualización}
    \begin{samepage} 
        \begin{twocolentry}{
            2020
        }
            \textbf{Udemy} Curso de Excel y PowerBI: análisis y visualización de datos
        \end{twocolentry}

        \vspace{0.2 cm}

        \begin{twocolentry}{
            2020
        }
            \textbf{Udemy} Excel completo: de principiante a avanzado
        \end{twocolentry}

        \vspace{0.2 cm}

        \begin{twocolentry}{
            2020
        }
            \textbf{EdX} Inclusión en el trabajo: habilidades para encontrar, mantener y progresar en un empleo.
        \end{twocolentry}

        \vspace{0.2 cm}

        \begin{twocolentry}{
            2023
        }
            \textbf{MathWorks} Simulink Onramp
        \end{twocolentry}       
        
        \vspace{0.2 cm}

        \begin{twocolentry}{
            2023
        }
            \textbf{MathWorks} MATLAB Onramp
        \end{twocolentry}
    \end{samepage}



\end{document}